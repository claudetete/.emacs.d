%% template from http://wiki.deimos.fr/Template_pour_créer_des_Cheat_Sheet_en_LaTeX
\documentclass[10pt,landscape]{article}
\usepackage[utf8x]{inputenc}  %Windows

\usepackage{multicol}
\usepackage{calc}
\usepackage{ifthen}
\usepackage[landscape]{geometry}
\usepackage{amsmath,amsthm,amsfonts,amssymb}
\usepackage{color,graphicx,overpic}
\usepackage{hyperref}

% PDF informations
\pdfinfo{
  /Title (cheatsheet_emacs.pdf)
  /Creator (LaTeX)
  /Author (Claude TETE)
  /Subject (Cheatsheet of GNU/Emacs)
  /Keywords (pdflatex, latex,pdftex,tex)}

% This sets page margins to .5 inch if using letter paper, and to 1cm
% if using A4 paper. (This probably isn't strictly necessary.)
% If using another size paper, use default 1cm margins.
\ifthenelse{\lengthtest { \paperwidth = 11in}}
    { \geometry{top=.5in,left=.5in,right=.5in,bottom=.5in} }
    {\ifthenelse{ \lengthtest{ \paperwidth = 297mm}}
        {\geometry{top=1cm,left=1cm,right=1cm,bottom=1cm} }
        {\geometry{top=1cm,left=1cm,right=1cm,bottom=1cm} }
    }

% Turn off header and footer
\pagestyle{empty}

%%% Redefine section commands to use less space
%%\makeatletter
%%\renewcommand{\section}{\@startsection{section}{1}{0mm}%
%%                                {-1ex plus -.5ex minus -.2ex}%
%%                                {0.5ex plus .2ex}%x
%%                                {\normalfont\large\bfseries}}
%%\renewcommand{\subsection}{\@startsection{subsection}{2}{0mm}%
%%                                {-1explus -.5ex minus -.2ex}%
%%                                {0.5ex plus .2ex}%
%%                                {\normalfont\normalsize\bfseries}}
%%\renewcommand{\subsubsection}{\@startsection{subsubsection}{3}{0mm}%
%%                                {-1ex plus -.5ex minus -.2ex}%
%%                                {1ex plus .2ex}%
%%                                {\normalfont\small\bfseries}}
%%\makeatother

% Don't print section numbers
\setcounter{secnumdepth}{0}

% Set vertical view instead of horizontal (set to 0 to let it choose)
\setcounter{unbalance}{0}

\setlength{\parindent}{0pt}
\setlength{\parskip}{0pt plus 0.5ex}

%My Environments
\newtheorem{example}[section]{Example}

% Dot lines between command and description
\def\cm#1#2{{\tt#1}\dotfill#2\par}

\begin{document}
\raggedright
\footnotesize
% Set number of columns
\begin{multicols}{3}


% multicol parameters
% These lengths are set only within the two main columns
%\setlength{\columnseprule}{0.25pt}
\setlength{\premulticols}{1pt}
\setlength{\postmulticols}{1pt}
\setlength{\multicolsep}{1pt}
\setlength{\columnsep}{2pt}

\begin{center}
     \Large{\underline{Cheatsheet GNU/Emacs}} \\
\end{center}


\section{Introduction}
Ces raccourcis sont pour cette configuration.
Tous les raccourcis sont sensibles à la casse. Quelques uns remplacent ceux par défaut...
Les raccourcis sont tous pensés pour le QWERTY !
J'utilise au moins les trois quarts d'entre eux, le reste est là comme pense-bête.

!!! Emacs 24 !!! les raccourcis pour coller dans la recherche incrémentale ont changé depuis Emacs 24 : C-y pour coller et M-y pour naviguer

\section{Explication}
C-a signifie ``Control + a''\\
C-a C-c signifie ``Control + a'' et ``Control + c''\\
C-a c signifie ``Control + a'' et ``c''\\
C-M-a signifie ``Control + Alt + a''\\
C est Control\\
M est Alt (Meta)\\
S est Shift\\
H est Menu (Hyper) next to right control\\
s est la touche Window (Super)\\






\section{Essentiel}
\cm{C-x C-c}{Quitter Emacs}
\cm{C-x z}{répète la dernière commande (continuer de répéter avec 'z')}
\cm{M-x}{Ouvre le minibuffer pour appeler une fonction}
\cm{C-TAB}{associé à M-TAB (qui est déjà utilisé par le window manager)}





\subsection{Déplacement}
\cm{C-M-HOME}{début de la fonction}
\cm{C-M-a}{début de la fonction}
\cm{C-M-END}{fin de la fonction}
\cm{C-M-e}{fin de la fonction}
\cm{C-M-u}{début du bloc}
\cm{C-M-d}{déplace dans un sous niveau de parenthèses}
\cm{C-M-f}{déplace d'un caractère ouvrant au caractère fermant}
\cm{C-M-b}{déplace d'un caractère ouvrant au caractère fermant}
\cm{M-e}{déplace à la fin d'une phrase}
\cm{C-c C-u}{déplace au début d'un bloc de preprocessing}
\cm{M-m}{déplace au début de la ligne avec l'indentation}
\cm{C-c C-w}{active la navigation de sous mot en sous mot dans le mode C (CommeCetExemple)}
\cm{H-LEFT}{aller à la parenthèse correspondante}
\cm{H-RIGHT}{aller à la parenthèse correspondante}
\cm{END}{aller à la fin de la ligne}
\cm{END END}{aller à la fin de l'écran}
\cm{END END END}{aller à la fin du buffer}
\cm{HOME}{aller au début de la ligne}
\cm{HOME HOME}{aller au début de l'écran}
\cm{HOME HOME HOME}{aller au début du buffer}
\cm{M-g}{aller à la ligne numero X}

\subsubsection{mineur Mode Colonne}
\cm{C-x C-n}{marque la colonne actuelle où le curseur se place quand il change de ligne}
\cm{C-u C-x C-n}{désactive le mode précédent}

\subsubsection{Ace Jump}
\cm{F9}{bouger le curseur par mot avec ace jump mode}
\cm{M-F9}{bouger le curseur par caractères avec ace jump mode}
\cm{C-F9}{bouger le curseur par ligne avec ace jump mode}
\cm{S-F9}{revenir à la position d'origine avec ace jump mode}





\subsection{Sélection}
\cm{M-h}{sélectionne tout le paragraphe courant}
\cm{C-M-h}{sélectionne toute la fonction courante}
\cm{C-M-@}{sélectionne tout le bloc de code source}
\cm{C-M-z}{sélectionner le mot sous le curseur}
\cm{C-ENTER}{sélectionner un rectangle avec CUA rectangle}

\subsubsection{Souris}
\cm{Left Click}{selection de texte}
\cm{Right Click}{agrandit une selection ou ouvre le menu contextuel}
\cm{Middle Click}{coller ou selectionner et ouvrir}





\subsection{Edition}
\cm{M-w}{copier la ligne ou la région}
\cm{C-w}{couper (kill) la ligne ou la région}
\cm{C-y}{Coller (yank)}
\cm{M-y}{Navigue dans la kill ring (après un yank) ou avec helm}
\cm{C-NUMBER M-w}{Copier dans un registre NUMBER}
\cm{C-NUMBER C-y}{Coller depuis un registre NUMBER}
\cm{C-S-SPACE}{insérer à ce point (global mark), chaque insertion de caractère et de copie sera inseré à ce point}
\cm{TAB}{indente correctement la ligne}
\cm{TAB TAB}{indente et cache/affiche un bloc}
\cm{C-o}{insère une nouvelle ligne sans bouger le curseur}
\cm{C-M-o}{insère une nouvelle ligne et l'aligne avec le curseur}
\cm{M-/}{complétion}
\cm{M-?}{plus de complétion}
\cm{M-(}{entoure la région de parenthèses}
\cm{S-ENTER}{insérer une ligne sans bouger le curseur}
\cm{H-ENTER}{joindre la ligne avec la suivante sans l'indentation}
\cm{M-ENTER}{insérer une nouvelle ligne suivante vide depuis n'importe où sur la ligne actuelle}
\cm{C-M-ENTER}{insérer une nouvelle ligne précédente vide depuis n'importe où sur la ligne actuelle}
\cm{C-c h}{afficher/cacher un bloc}
\cm{C-M-l}{indente la fonction}
\cm{C-j}{insère une nouvelle ligne sans bouger}

\subsubsection{Remplacer}
\cm{M-r}{remplacer la string dans le buffer ou la région}
\cm{M-x flush-lines}{supprime toutes les lignes correspondantes à une expression rationnelle}
\cm{M-x keep-lines}{garde seulement les lignes correspondantes à une expression rationnelle}

\subsubsection{Casse}
\cm{M-u}{passe en majuscule la suite du mot ou région}
\cm{M-l}{passe en minuscule la suite du mot ou région}
\cm{M-c}{passe en majuscule le premier caractère et en minuscule les autres pour mot ou région}
\cm{M-- SHORTCUT}{utilisé avec les précédents raccourcis pour appliquer au précédent mot et non au suivant}
\cm{C-x C-l}{passe en minuscule la région sélectionnée}
\cm{C-x C-u}{passe en majuscule la région sélectionnée}


\subsubsection{Fill}
\cm{C-c ]}{justifier le paragraphe}
\cm{C-c [}{justifier la région ou la ligne}

\subsubsection{Rectangle}
\cm{C-x r r}{copier une région/rectangle dans un registre}
\cm{C-x r i}{insérer un registre comme un rectangle}
\cm{C-x r t}{remplacer une région/rectangle par une chaîne de caractère}
\cm{C-x r o}{insérer un rectangle d'espace}
\cm{C-x r d}{supprime la région/rectangle}
\cm{C-ENTER}{sélectionner un rectangle}
\cm{M-m}{copier un rectangle comme une copie normale (CUA)}
\cm{M-s}{remplir le rectangle avec une string (CUA)}
\cm{M-r}{remplacer avec regexp dans un rectangle (CUA)}
\cm{M-i}{incrémenter le premier nombre de chaque ligne (CUA)}
\cm{M-n}{insérer une suite de nombre (CUA)}
\cm{M-'}{limiter aux lignes non vide (CUA)}
\cm{M-'}{limiter aux lignes qui correspondent à la regexp (CUA)}
\cm{M-a}{aligner les mots à gauche (CUA)}

\subsubsection{Échanger}
\cm{M-t}{échange le mot courant avec le suivant}
\cm{C-M-t}{échange le mot courant avec le suivant (\_ est considéré comme un séparateur)}
\cm{C-x C-t}{échange la ligne courante avec la précédente}
\cm{C-t}{échange le caractère courant avec le précédent}

\subsubsection{Supprimer}
\cm{C-x C-o}{supprime toutes les lignes vides et en laisse une seule}
\cm{M-SPACE}{supprime tous les espaces et en laisse un}
\cm{C-k}{supprime le reste de la ligne}
\cm{C-S-BACKSPACE}{supprime toute la ligne}
\cm{M-d}{supprimer mot en avant (doit être au début)}
\cm{C-z}{supprimer mot en arrière (doit être à la fin)}
\cm{C-(}{supprimer parenthèses qui correspondent}
\cm{H-/}{commenter or décommenter la région ou la ligne}
\cm{M-DEL}{supprime le mot suivant}
\cm{C-DEL}{supprime la fin de la ligne}
\cm{C-BACKSPACE}{supprime le debut de la ligne}
\cm{M-z}{supprime jusqu'à un caractère spécifié}



\subsection{Annuler}
\cm{M-\_}{annuler infini}
\cm{C-/}{annuler (undo tree mode)}
\cm{C-?}{refaire (undo tree mode)}
\cm{C-x u}{afficher arbre d'annuler}
\cm{UP/DOWN}{se déplacer dans l'arbre d'annuler et voir les changement dans le buffer (undo tree mode)}
\cm{t}{afficher ou cacher l'heure (undo tree mode)}
\cm{q}{quitter (undo tree mode)}
\cm{ENTER}{quitter (undo tree mode)}
\cm{PAGEUP/PAGEDOWN}{deplace le curseur de 10 annuler/refaire (undo tree mode)}
\cm{C-UP}{va au précédant noeud (undo tree mode)}
\cm{C-DOWN}{va au noeud suivant (undo tree mode)}





\subsection{Fichiers}
\cm{C-x C-f}{ouvrir un fichier (avec helm si utilisé)}
\cm{C-x C-q}{passe de lecture seule à lecture et écriture (et inversement)}
\cm{C-x ENTER f dos ENTER}{sauvegarde le buffer actuel avec des CRLF comme saut de ligne (\textbackslash n)}
\cm{C-x ENTER f unix ENTER}{sauvegarde le buffer actuel avec des LF comme saut de ligne (\textbackslash n)}
\cm{C-x ENTER r dos ENTER}{relis le buffer actuel avec des CRLF comme saut de ligne (\textbackslash r\textbackslash n)}
\cm{C-x ENTER r unix ENTER}{relis le buffer actuel avec des LF comme saut de ligne (\textbackslash n)}
\cm{C-c f}{trouver un fichier dans le projet avec regex (gtags)}
\cm{M-f}{trouver un fichier par son nom avec grep et affiche la liste dans dired}
\cm{M-p}{lit de nouveau le fichier (revert-buffer)}
\cm{C-`}{passer de fichier header à source et inversement}
\cm{C-F4}{passer de fichier header à source et inversement}

\subsubsection{Bookmarks}
\cm{C-x r m}{ajoute en bookmark le buffer actuel}
\cm{C-x r l}{affiche une liste des bookmarks}
\cm{d}{marqué comme à supprimer (dans la liste des bookmarks)}
\cm{e}{édite les notes (dans la liste des bookmarks)}
\cm{a}{affiche les notes (dans la liste des bookmarks)}
\cm{x}{applique tous les changements (dans la liste des bookmarks)}
\cm{C-c b}{afficher une liste de bookmarks}
\cm{C-c v}{ajoute le buffer actuel dans les bookmarks}



\subsection{Buffer}
\cm{C-x n n}{n'affiche que la région sélectionnée (on ne peut plus interagir avec le reste du buffer)}
\cm{C-x n w}{sortir du mode précédent (narrow)}
\cm{F4}{fermer le buffer et la window actuelle}
\cm{M-`}{fermer le buffer actuel}
\cm{C-x C-b}{afficher une liste de buffers}
\cm{H-UP}{scroll up l'affichage}
\cm{H-DOWN}{scroll down l'affichage}
\cm{C-c f}{ouvrir un fichier recent (avec helm si utilisé)}

\subsubsection{Special}
\cm{M-2}{aller au buffer de grep ou de ack}
\cm{M-3}{aller au buffer de compilation}
\cm{M-4}{aller au buffer de vc ou vc-diff}
\cm{M-5}{aller au buffer de occur}
\cm{M-6}{aller au buffer de aide}





\subsection{Windows}
\cm{C-l}{recentrer le buffer dans la window au niveau du curseur}
\cm{C-l C-l}{recentrer buffer dans la window avec le curseur en haut}
\cm{C-l C-l C-l}{recentrer buffer dans la window avec le curseur en bas}
\cm{C-x r f}{garde en mémoire (registre) la configuration actuelle des windows}
\cm{C-x r j}{restaure une configuration de windows depuis un registre}
\cm{C-x +}{équilibre la taille des windows}
\cm{S-UP/DOWN/LEFT/RIGHT}{va dans la fenêtre donnée par la direction}

\subsubsection{Division}
\cm{C-x 2}{divise horizontalement en deux la window actuelle}
\cm{C-x 3}{divise verticalement en deux la window actuelle}
\cm{C-x 1}{n'affiche que la window actuelle}
\cm{C-x 0}{ferme la window actuelle}
\cm{C-x 5 2}{affiche la window actuelle dans une fenêtre externe}
\cm{C-x 5 0}{ferme la fenêtre externe}
\cm{M-NumpadPoint}{supprime la window}
\cm{M-Numpad0}{supprime toutes les window exceptée une}
\cm{M-Numpad2}{divise horizontalement en deux la window actuelle et va dans celle du dessous}
\cm{M-Numpad6}{divise verticalement en deux la window actuelle et va dans celle de droite}
\cm{M-Numpad5}{affiche la window actuelle dans une fenêtre externe et la maximise}
\cm{M-NumpadENTER}{ferme la fenêtre externe}

\subsubsection{Taille}
\cm{C-S-UP}{agrandir verticalement la window actuelle}
\cm{C-S-DOWN}{réduire verticalement la window actuelle}
\cm{C-S-LEFT}{agrandir horizontalement la window actuelle}
\cm{C-S-RIGHT}{réduire horizontalement la window actuelle}





\subsection{Recherche}

\subsubsection{Incrémentale}
\cm{C-s}{recherche incrémentale en avant}
\cm{C-s C-s}{recherche incrémentale en avant avec dernière recherche}
\cm{C-r}{recherche incrémentale en arrière}
\cm{C-r C-r}{recherche incrémentale en arrière avec dernière recherche}
\cm{C-M-s}{recherche incrémentale en avant avec regex}
\cm{C-M-s C-M-s}{recherche incrémentale en avant avec regex et dernière recherche}
\cm{C-M-r}{recherche incrémentale en arrière avec regex}
\cm{C-M-r C-M-r}{recherche incrémentale en arrière avec regex et dernière recherche}
\cm{M-s w}{recherche incrémentale du mot suivant}
\cm{C-u C-SPACE}{après une recherche incrémentale, revenir au début}
\cm{M-\%}{pendant une recherche incrémentale démarre un remplacement}
\cm{C-M-v}{recherche incrémentale du mot sous le curseur}
\cm{C-o}{lister toutes les occurrences dans le buffer actuel lorsqu'on est dans une recherche incrémentale}

\subsubsection{Occur}
\cm{M-s o}{affiche toutes les occurrences dans le buffer courant}
\cm{C-c e}{lister toutes les occurrences dans le buffer actuel}
\cm{C-M-c}{lister toutes les occurrences du mot dans le buffer actuel}

\subsubsection{Grep/Ack}
\cm{C-F3}{recherche un mot avec ack}
\cm{M-F3}{recherche dans les même type de fichiers avec ack}
\cm{F3}{match/erreur suivante}
\cm{S-F3}{match/erreur précédente}

\subsubsection{Minibuffer}
\cm{M-s}{Recherche avec regex dans l'historique des commandes depuis la première commande}
\cm{M-r}{Recherche avec regex dans l'historique des commandes depuis la dernière commande}





\subsection{Register}
\cm{C-c i}{insère le texte d'un registre}
\cm{C-x r SPACE}{marque dans un registre ou bookmark un emplacement dans un buffer (doit être nommé)}
\cm{C-x r j}{va à l'emplacement marqué dans un registre ou bookmark (par nom)}
\cm{C-x r s}{sauve une région selectionnée dans un registre}
\cm{C-x r i}{insère le texte d'un registre}
\cm{M-x append-to-register}{insère la région selectionnée à la fin d'un registre}
\cm{M-x prepend-to-register}{insère la région selectionnée au début d'un registre}





\subsection{Macro}
\cm{C-x (}{enregistrer une macro}
\cm{C-x )}{arrêter l'enregistrement d'une macro}
\cm{C-x C-k r}{lance une macro sur une région}
\cm{C-x e}{lance une macro}
\cm{C-x q}{ajoute une pause pour demander a l'utilisateur de continuer (pendant l'enregistrement d'une macro)}
\cm{F8}{lancer la dernière macro}
\cm{S-F8}{enregister/arrêter l'enregistrment d'une macro}
\cm{C-F8}{nommer la dernière macro}
\cm{M-F8}{editer la dernière macro}
\cm{H-F8}{sélectionner la macro précédente}
\cm{H-S-F8}{sélectionner la macro suivante}





\subsection{Highlight}
\cm{C-x w r}{désactiver un surlignement avec regex}





\subsection{Info}
\cm{M-=}{information à propos de la région sélectionnée}
\cm{C-x =}{information à propos d'un caractère}
\cm{C-u C-x =}{information à propos du texte (face/couleur/style...)}
\cm{M-L}{numéro de ligne}
\cm{C-h e}{affiche ``*Messages*''}
\cm{M-x list-color-display}{afficher une liste des couleurs}

\subsubsection{Aide}
\cm{C-h a}{rechercher dans l'aide avec regex}
\cm{C-h k}{donner l'aide à propos d'un raccourci}
\cm{C-h w}{donner l'aide à propos d'une fonction}
\cm{C-h v}{donner l'aide à propos d'une variable}
\cm{C-h f}{donner l'aide à propos d'une fonction}
\cm{C-c C-b}{sujet précédent}
\cm{C-c C-f}{sujet suivant}
\cm{s}{recherche}
\cm{t}{sujet parent}
\cm{l}{précédent}






\section{Mode}

\subsection{ECB}
\cm{M-1}{afficher ou cacher la window d'ECB}
\cm{F2 (ou bouton suivant)}{afficher ou cacher la window de compilation d'ECB}
\cm{C-c w}{alterner entre deux taille de la window d'ECB}
\cm{M-q}{aller à la window ECB des dossiers}
\cm{M-q M-q}{aller à la window ECB des dossiers et la maximise}
\cm{M-a}{aller à la window ECB des sources}
\cm{M-a M-a}{aller à la window ECB des sources et la maximise}
\cm{M-z}{aller à la window ECB historique}
\cm{M-z M-z}{aller à la window ECB historique et la maximise}
\cm{M-\textbackslash}{aller à la window ECB fonctions}
\cm{M-\textbackslash M-\textbackslash}{aller à la window ECB fonctions et la maximise}
\cm{ENTER}{aller à la window ECB source quand on est dans la windows ECB dossiers}
\cm{M-ENTER}{aller à la window ECB fonctions quand on est dans la window ECB sources}
\cm{ESCAPE}{retourne aux windows d'édition (en dehors d'ECB)}





\subsection{Langages}
\cm{C-M-\textbackslash}{indenter une sélection}
\cm{M-;}{insérer un commentaire à la bonne place}
\cm{C-c C-w}{alterner entre activer et désactiver le mode sous mot ( dans le mode C)}
\cm{C-c @ C-d}{cacher un bloc de preprocessing}
\cm{C-c @ C-s}{afficher un bloc caché de préprocessing}
\cm{F10}{appeler la compilation}

\subsubsection{C}
\cm{M-LEFT}{cache/affiche un bloc (folding)}
\cm{M-RIGHT}{cache/affiche un bloc (folding)}
\cm{M-UP}{cache tous les blocs}
\cm{M-DOWN}{affiche tous les blocs}





\subsection{CEDET/Semantic}
\cm{C-c , ,}{Parser le buffer actuel}
\cm{C-c , j}{va au symbole sous le curseur (voir M-. dans custom)}
\cm{C-c , u}{va au fichier inclus}
\cm{C-c , g}{trouve toutes les références du symbole sous le curseur}
\cm{C-c , G}{trouve tous les appels à la fonction actuelle}





\subsection{Tags (GNU Global, Semantic, Etags)}
\cm{M-.}{aller à la définition (semantic)}
\cm{C-\textgreater}{aller à la définition (semantic)}
\cm{S-CLICK\_GAUCHE}{aller à la définition (semantic)}
\cm{C-\textless}{revenir de la définition (semantic)}
\cm{CLICK\_BACK}{revenir de la définition (semantic)}
\cm{C-M-.}{rechercher avec grep dans les tags (gtags)}
\cm{RIGHT\_CLICK}{affiche un menu avec la listes des fonctions, variables et include}
\cm{C-x C-SPACE}{revenir en arrière après avoir utilisé une recherche de tags}
\cm{C-c C-f}{affiche un menu avec la listes des fonctions, variables et include avec helm}





\subsection{RTRT script}
\cm{C-c p o}{aligner 'init' pour RTRT}
\cm{C-c p ;}{aligner 'ev' pour RTRT}
\cm{C-c p [}{aligner 'init' et 'ev' pour RTRT}
\cm{C-c p s}{aligner les stub}
\cm{C-c r s}{supprime tous les espaces avant une virgule}
\cm{M-LEFT}{aller vers le précédent test}
\cm{M-RIGHT}{aller vers le test suivant}
\cm{M-UP}{aller vers le haut vers SERVICE}
\cm{M-DOWN}{aller vers le bas vers END SERVICE}
\cm{M-PAGEUP}{cache tous les blocs}
\cm{M-PAGEDOWN}{affiche tous les blocs}





\subsection{AUCTeX}
\cm{F10}{Sauve er compile le buffer (utile pour la preview pdf)}





\subsection{Orthographe}
\cm{F7}{correction orthographique sur le buffer}
\cm{S-F7}{correction orthographique sur le mot sous le curseur}
\cm{M-F7}{correction orthographique sur la région sélectionnée}
\cm{C-F7}{correction orthographique sur les commentaires et string dans le buffer}
\cm{C-M-F7}{changer le dictionnaire}
\cm{C-c \$}{changer le dictionnaire}





\subsection{Browse Kill Ring}
\cm{UP}{précédent}
\cm{DOWN}{suivant}
\cm{C-g}{quitter}
\cm{ESCAPE}{quitter}
\cm{F2}{quitter (si ecb est utilisé)}
\cm{y}{insérer}
\cm{i}{insérer}
\cm{ENTER}{insérer et quitter le mode}
\cm{u}{insérer, bouger et quitter}
\cm{o}{insérer et bouger}
\cm{q}{quitter}
\cm{d}{supprimer le texte killé actuel}
\cm{s}{recherche en avant}
\cm{r}{recherche en arrière}
\cm{e}{éditer}
\cm{n}{suivant}
\cm{p}{précédent}
\cm{g}{mettre à jour}
\cm{l}{recherche avec occur}
\cm{h}{aide}
\cm{MIDDLE CLICK}{insérer}





\subsection{Alignement}
\cm{C-c p p}{aligner une sélection avec les règles du mode actuel}
\cm{C-c p l}{aligner une sélection avec regex}
\cm{C-c p =}{aligner =, ==, +=, *= etc}





\subsection{Version Control}
\cm{C-x v v}{faire la prochaine action logique sur le fichier}
\cm{C-x v =}{diff}
\cm{C-x v l}{afficher l'historique}
\cm{C-x v i}{ajoute comme nouveau fichier}
\cm{C-x v +}{mettre à jour}
\cm{C-x v u}{annuler un checkout}
\cm{C-x v ~}{afficher d'autres version}
\cm{C-x v d}{afficher tous les fichier qui ne sont à jour}
\cm{C-x v g}{retracer les éditions et ajout pour chaque ligne}
\cm{C-x v C-h}{aide sur les commandes du gestionnaire de version}

\subsubsection{Subversion}
\cm{q}{quitter et supprime la window de diff ou de log}

\subsubsection{ClearCase}
\cm{C-c c c}{gui checkout}
\cm{C-c c =}{gui diff with previous version}
\cm{C-c c l}{gui history}
\cm{C-c c u}{gui uncheckout}
\cm{C-c c L}{gui version tree}
\cm{C-c c e}{gui explorer}
\cm{C-c c v}{gui version properties}
\cm{C-c c p}{gui element properties}
\cm{C-c c i}{gui checkin}
\cm{C-c c f}{gui find checkout}
\cm{C-c c s}{editer le config spec}
\cm{C-c C-c}{sauve et quitte l'édition de config spec}





\subsection{Calculatrice}
\cm{M-Numpad*}{afficher la calculatrice rapide dans le minibuffer}
\cm{C-x * c}{afficher la calculatrice (notation polonaise)}
\cm{C-x * *}{afficher la calculatrice avec le dernier mode utilisé}
\cm{C-x * q}{afficher la calculatrice rapide dans le minibuffer}
\cm{C-x * r}{ajoute une région/rectangle dans la calculatrice}
\cm{C-x * :}{ajoute une région/rectangle et somme les colonnes}
\cm{C-x * \_}{ajoute une région/rectangle et somme les lignes}
\cm{C-x * x}{quitter la calculatrice}
\cm{C-x * e}{ajouter une région, calculer et coller le résultat}
\cm{y}{coller dans le buffer}
\cm{\$}{dernier résultat}
\cm{16\#}{préfixe pour l'hexadécimale}
\cm{8\#}{préfixe pour l'octale}
\cm{2\#}{préfixe pour le binaire}
\cm{d r}{change l'affichage dans la base voulu}





\subsection{Calendrier}
\cm{M-x calendar}{afficher le calendrier}
\cm{d}{afficher l'agenda pour la date sélectionnée}
\cm{S}{donner les heures de lever et couché de soleil}
\cm{M}{donner les phases de la lune}





\subsection{Dired}
\cm{C-x C-q}{éditer nom ou droits de fichiers}
\cm{C-c C-c}{appliquer les modifications du mode précédent}
\cm{\% R}{renommer des fichiers avec les expression rationnelles}
\cm{H-ENTER}{ouvrir avec l'application par défaut dans MS Windows}
\cm{[}{dossier parent (même touche que \^ sur de l'azerty)}





\subsection{Web}
Chaque raccourci ouvre le navigateur web.\\
\cm{F12}{traduire mot ou région anglais-français (wordreference.com)}
\cm{M-F12}{traduire mot ou région français-anglais (wordreference.com)}
\cm{C-F12}{conjuguer mot ou région (leconjugueur.com)}
\cm{S-F12}{trouver synonymes français du mot ou région (synonymes.com)}
\cm{F11}{rechercher mot ou région (en.wikipedia.org)}
\cm{M-F11}{rechercher mot ou région (fr.wikipedia.org)}
\cm{C-F11}{rechercher mot ou région (google.com}
\cm{C-M-F11}{rechercher mot ou région (google.fr)}




\end{multicols}

\end{document}
